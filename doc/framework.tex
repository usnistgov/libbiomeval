%
% Framework API
%
\chapter{Framework}
\label{chp-framework}

The \namespace{Framework} package is used to retrieve information about the
\lname itself, as well as to provide services through general purpose utility
functions to other parts of the framework.

\section{Versioning}
Version numbers, the compiler used, and other framework metadata
can be queried by applications.  Versioning information is recorded in the
\sname \code{Makefile} and populated in the function implentation at
compile-time.

\begin{lstlisting}[caption={Using the \namespace{Framework} API}, label=frameworkuse]
/* "Framework Version: 0.4" */
std::cout << "Framework Version: " << BE::Framework::getMajorVersion() << "." <<
    BE::Framework::getMinorVersion() << std::endl;

/* "Compiler Used: clang v5.1.0" */
std::cout << "Compiler Used: " << BE::Framework::getCompiler() << " v" <<
    BE::Framework::getCompilerVersion() << std::endl;

/* "Date/Time Compiled: Jan 24 2014 12:16:01" */
std::cout << "Date/Time Compiled: " << BE::Framework::getCompileDate() << " " <<
    BE::Framework::getCompileTime() << std::endl;
\end{lstlisting}

\section{Enumerations}
\label{sec_framework-enumerations}

As of \CppXI, \code{enum}s can be strongly-typed. The \lname makes use of
these strongly-typed \code{enum class}es throughout. As an added convenience,
functions converting to and from \code{enum}s, \code{string}s, and \code{int}s
are implicitly implemented easily via a \code{template}, eliminating many lines
of boiler-plate code and creating equivalence in functionality among
\code{enum class}es throughout \sname.

At the core of \namespace{Framework::Enumeration} is a \code{const} mapping of
\code{enum} to \code{string}, defined by you in code and instantated at
compile-time. As demonstrated in \lstref{framework-enumeration}, simply define
your \code{enum class} and populate the \code{map}.

\begin{lstlisting}[caption={\namespace{Framework::Enumeration}}, label=framework-enumeration]
/*
 * color.h
 */

enum class Color
{
	Black,
	Blue,
	Green
};

/*
 * color.cpp
 */

#include <be_framework_enumeration.h>

template<>
const std::map<Color, std::string>
BiometricEvaluation::Framework::EnumerationFunctions<Color>::enumToStringMap {
	{Color::Black, "Black"},
	{Color::Black, "Blue"},
	{Color::Green, "Green"}
};

/*
 * application.cpp
 */

#include <color.h>

/* "Black" */
std::cout << to_string(Color::Black) << std::endl;
/* "2" */
std::cout << to_int_type(Color::Green) << std::endl;
/* Color::Blue */
Color color = to_enum<Color>("Blue");
\end{lstlisting}

While \namespace{Framework::Enumeration} was created for \sname, the
\code{template}'s only dependency is \class{Exception}, and so it can easily be
used in other \CppXI projects.
