%
% Framework API
%
\chapter{Framework}
\label{chp-framework}

The \namespace{Framework} package is used to retrieve information about the
\lname itself, as well as to provide services through general purpose utility
functions to other parts of the framework.

\section{Versioning}
Version numbers, the compiler used, and other framework metadata
can be queried by applications.  Versioning information is recorded in the
\sname \code{Makefile} and populated in the function implementation at
compile-time.

\begin{lstlisting}[caption={Using the \namespace{Framework} API}, label=frameworkuse]
/* "Framework Version: 0.4" */
std::cout << "Framework Version: " << BE::Framework::getMajorVersion() << "." <<
    BE::Framework::getMinorVersion() << std::endl;

/* "Compiler Used: clang v5.1.0" */
std::cout << "Compiler Used: " << BE::Framework::getCompiler() << " v" <<
    BE::Framework::getCompilerVersion() << std::endl;

/* "Date/Time Compiled: Jan 24 2014 12:16:01" */
std::cout << "Date/Time Compiled: " << BE::Framework::getCompileDate() << " " <<
    BE::Framework::getCompileTime() << std::endl;
\end{lstlisting}

\section{Enumerations}
\label{sec_framework-enumerations}

As of \CppXI, \code{enum}s can be strongly-typed. The \lname makes use of
these strongly-typed \code{enum} classes throughout. As an added convenience,
functions converting to and from \code{enum}s, \code{string}s, and \code{int}s
are defined by using a \code{template}, eliminating many lines
of boiler-plate code and creating equivalence in functionality among
\code{enum class}es throughout \sname. The output stream operator \code{<<} is
also defined by the template.

At the core of \namespace{Framework::Enumeration} is a \code{const} mapping of
\code{enum} to \code{string}, defined in code and instantiated at
compile-time. The procedure to create a enum-to-string map is as follows:
\begin{itemize}
\item Include the \code{be\_framework\_enumeration.h} file to access the
template definitions;
\item Define the \code{enum} class;
\item Use the \code{BE\_FRAMEWORK\_ENUMERATION\_DECLARATIONS} macro to declare
the enum-to-string map;
\item Define the map from the \code{enum} elements to \code{std::string} objects;
\item Use the \code{BE\_FRAMEWORK\_ENUMERATION\_DEFINITIONS} macro to define the
functions based on the map (\code{to\_string}, etc.).
\end{itemize}

This procedure is demonstrated in \lstref{framework-enumeration}.
The functions defined by the template exist within the 
\namespace{BiometricEvaluation::Framework::Enumeration} namespace. In the
example application, the stream operator is used both with a call to the
\code{to\_string} function as well as directly. Typically the former where a
stream operation is unavailable, calling a C program for example.

\begin{lstlisting}[caption={\namespace{Framework::Enumeration}}, label=framework-enumeration]
/*
 * color.hpp
 */
#include <be_framework_enumeration.h>
enum class Color
{
        Black,
        Blue,
        Green
};
BE_FRAMEWORK_ENUMERATION_DECLARATIONS(
    Color, Color_EnumToStringMap);

/*
 * color.cpp
 */
#include "tfr.h"

using namespace BiometricEvaluation::Framework::Enumeration;
        
const std::map<Color, std::string>
Color_EnumToStringMap = {
        {Color::Black, "Black"},
        {Color::Blue, "Blue"},
        {Color::Green, "Green"}
};

BE_FRAMEWORK_ENUMERATION_DEFINITIONS(
        Color,
        Color_EnumToStringMap);

/*
 * Application
 */
#include <iostream>
int main()
{
        std::cout << to_string(Color::Black) << std::endl;
        std::cout << Color::Black << std::endl;
        std::cout << to_int_type(Color::Green) << std::endl;
        Color color = to_enum<Color>("Blue");
        std::cout << color << std::endl;
}
\end{lstlisting}

While \namespace{Framework::Enumeration} was created for \sname, the
\code{template}'s only dependency is \class{Exception}, and so it can easily be
used in other \CppXI projects.
