%
% Face API
%
\chapter{Face}
\label{chp-face}
The \namespace{Face} package provides access to facial information stored in
standard record formats. Within this package are defined the common elements
relevant to facial images, such as hair color, expression, pose angle, and
others.

\subsection{ISO/INCITS Face Views}
The \class{Face::INCITSView} class, extends \class{View::View}
(See~\ref{chp-view}) by adding methods to retrieve facial information. A
\class{Face::INCITSView} object cannot be constructed by applications but
rather this class is subclassed to represent each standard format.
For example, the \class{ISO2005View} class represents the 
ISO/IEC 19794-5~\cite{std:ansi385-2004} standard.

\lstref{lst:isofaceuse} shows how an application can query the information from
a standard ISO/INCITS-385 facial information record.

\begin{lstlisting}[caption={Using the Face::ISO2005View Class}, label=lst:isofaceuse]
#include <iostream>
#include <iomanip>
#include <be_face_iso2005view.h>

using namespace std;
using namespace BiometricEvaluation;
using namespace BiometricEvaluation::Framework::Enumeration;

void
printViewInfo(View::View &view)
{
	/*
	 * Provided by the View::View interface.
	 */
	cout << "Image resolution is " << view.getImageResolution() << endl;
	cout << "Scan resolution is " << view.getScanResolution() << endl;
	cout << "Image size is " << view.getImageSize() << endl;
	cout << "Image depth is " << view.getImageColorDepth() << endl;
	cout << "Compression is " <<
	    view.getCompressionAlgorithm() << endl;

	try {
		std::shared_ptr<Image::Image> theImage = view.getImage();
		cout << "Information from the Image data item:" << endl;
		cout << "\tResolution: " << theImage->getResolution() << endl;
		cout << "\tDimensions: " << theImage->getDimensions() << endl;
		cout << "\tDepth: " << theImage->getColorDepth() << endl;
	} catch (Error::Exception &e) {
		cout << "Caught " << e.what() << endl;
	}
	cout << "------------------------------------------" << endl;
}

void
printFaceInfo(Face::ISO2005View &facev)
{
	/*
	 * Provided by the Face::INCITSView interface.
	 */
	cout << "Gender: " << facev.getGender() << endl;
	cout << "Eye Color: " << facev.getEyeColor() << endl;
	cout << "Hair Color: " << facev.getHairColor() << endl;
	cout << "Expression: " << facev.getExpression() << endl;

	Face::PoseAngle pa =  facev.getPoseAngle();
	cout << "Pose angle info: ";
	cout << "Yaw/Uncer: " << (int)pa.yaw << "/" << (int)pa.yawUncertainty;
	cout << "; Pitch/Uncer: "
	    << (int)pa.pitch << "/" << (int)pa.pitchUncertainty;
	cout << "; Roll/Uncer: "
	    << (int)pa.roll << "/" << (int)pa.rollUncertainty << endl;

	cout << "Image type is " << facev.getImageType() << endl;
	cout << "Image data type is " << facev.getImageDataType()
	    << endl;
	cout << "Color space is " << facev.getColorSpace() << endl;
	cout << "Source type is " << facev.getSourceType() << endl;
	cout << "Device type is " << "0x" << hex << setw(4) << setfill('0')
	    << facev.getDeviceType() << dec << endl;

	Face::PropertySet properties;
	bool haveProps = facev.propertiesConsidered();
	if (haveProps) {
		facev.getPropertySet(properties);
		cout << "There are " << properties.size() << " properties: ";
		for (size_t i = 0; i < properties.size(); i++) {
			if (i != properties.size() - 1)
				cout << properties[i] << ", ";
			else
				cout << properties[i];
		}
		cout << endl;
	} else {
		cout << "There are no properties." << endl;
	}

	Feature::MPEGFacePointSet fps;
	facev.getFeaturePointSet(fps);
	cout << "There are " << fps.size() << " feature points." << endl;
	if (fps.size() != 0) {
		cout << "\tType\tCode\tPosition" << endl;
	}
	for (size_t i = 0; i < fps.size(); i++) {
		cout << "\t" << (int)fps[i].type
		    << "\t" << (int)fps[i].major << "." << (int)fps[i].minor
		    << "\t" << fps[i].coordinate
		    << endl;
	}
	cout << "------------------------------------------" << endl;
}

int
main(int argc, char* argv[])
{
	Face::ISO2005View facev;
	try {
		facev = Face::ISO2005View("test_data/face01.iso2005", 1);
	} catch (Error::Exception &e) {
		cout << "Caught " << e.what()  << endl;
		return (EXIT_FAILURE);
	}
	printViewInfo(facev);
	printFaceInfo(facev);
	return(EXIT_SUCCESS);
}
\end{lstlisting}
