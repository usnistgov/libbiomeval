%
% View APIs
%
\chapter{View}
\label{chp-view}
Within the \lname a view represents all the information that was derived
from an image of a biometric sample. For example, with a fingerprint image, any
minutiae that were extracted from that image, as well as the image itself,
are contained within a single \class{View} object. In many cases the image may
not be present, however the image size and other information is contained within
a biometric record, along with the derived information. A view is used
to represent these records as well.

In the case where a raw image is part of the biometric record, the \class{View} 
object's related \class{Image}~(\chpref{chp-image} object will have identical
size,
resolution, etc. values because the \class{View} class sets the \class{Image}
attributes directly.
For other image types (e.g. \class{JPEG}) the \class{Image} object will return
attribute values taken from the image data.

Views are high-level abstractions of the biometric sample, and concrete
implementations of a \class{View} include finger, face, iris, etc. views based
on a specific type of biometric. Therefore, \class{View} objects are not
created directly. Subclasses, such as finger views (see~\chpref{chp-finger}),
represent the specific type of biometric sample.

\class{View} objects are created with information taken from a biometric data
record, an ANSI/NIST 2007 file, for example. Most record formats contain 
information about the image itself, such as the resolution and size. The object
can be used to retrieve this information. However, the data may differ from that
contained in the image itself, and applications can compare the corresponding
values taken from the \class{Image} object (when available) to those taken from 
the \class{View} object.

\lstref{lst:viewuse} shows a function that will print the information obtained
from any \class{View} object.

\begin{lstlisting}[caption={View::View Class}, label=lst:viewuse]
void
printViewInfo(BiometricEvaluation::View::View &view)
{
        cout << "Image size is " << view.getImageSize() << endl;
        cout << "Image resolution is " << view.getImageResolution() << endl;
        cout << "Scan resolution is " << view.getScanResolution() << endl;
        cout << "Image color depth is " << view.getImageColorDepth() << endl;
        cout << "Compression is " << view.getCompressionAlgorithm() << endl;
        try {
                auto theImage = view.getImage();
                cout << "Information from the Image data item:" << endl;
                cout << "\tResolution: " << theImage->getResolution() << endl;
                cout << "\tDimensions: " << theImage->getDimensions() << endl;
                cout << "\tDepth: " << theImage->getColorDepth() << endl;
        } catch (Error::Exception &e) {
                cout << "Caught " << e.what() << endl;
        }
}
\end{lstlisting}

\section{ANSI/NIST Views}
The ANSI/NIST standard~\cite{std:an2k} describes fixed and variable resolution
finger, latent, and palm image records. These are represented within \sname
by \class{View::AN2KView} (subclass of \class{View::View} and\\
\class{View::AN2KViewVariableResolution}, subclass of \class{AN2KView}.
As these classes only define the common interface for the ANSI/NIST records,
objects of these class types cannot be created. These classes are further
extended by classes in the \namespace{Finger}, \namespace{Latent}, and
\namespace{Palm} name spaces. See~\ref{sec-an2kminutiaedatarecord} and
\ref{sec-an2kpalmview}.
