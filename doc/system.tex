%
% System API
%
\chapter{System}
\label{chp-system}
The \namespace{System} package provides a set of functions that
return information about the hardware and operating system. This information
can be used by applications to determine the amount of real memory, number of
central processing units, or current load average, and can be
used to dynamically tailor the application behavior, or simply to provide
additional information in a runtime log.

\lstref{lst:cpucountuse} shows how an application can spawn several child
processes based on the number of CPUs and memory available. Note that this
information may not be available on all platforms, and therefore, the
application must be prepared to handle that situation.

\begin{lstlisting}[caption={Using the \namespace{System} CPU Count Information}, label=lst:cpucountuse]
#include <iostream>
#include <be_system.h>

using namespace BiometricEvaluation;

int
main(int argc, char* argv[]) {

    // perform some application setup ...

    uint32_t cpuCount;
    uint64_t memSize, vmSize;
    try {
        cpuCount =  System::getCPUCount();
        cpuCount--;    // subtract one CPU for the parent process
        memSize = System::getRealMemorySize();
	uint64_t vmSize;
	Process::Statistics stats{};
	std::tie(std::ignore, vmSize, std::ignore, std::ignore, std::ignore)
	    = stats.getMemorySizes();
        memSize -= vmSize;   // subtract off memory used by parent

        // Give each child a fraction of the memory
        spawnChildren(cpuCount, memSize / cpuCount);
    } catch (Error::NotImplemented) {
            std::cout << "Running a single process only." << endl;
    }

    // processing done by parent ...
}

\end{lstlisting}

\subsection{MemoryLogger}
\label{sec-memorylogger}
The \class{MemoryLogger} is used to periodically add a log entry to an
existing \class{Logsheet}, each entry containing system memory usage. Upon
construction logging will be off and must be started by the application,
specifying the logging interval.

\begin{lstlisting}[caption={Using the \namespace{System::MemoryLogger} Class}, label=lst:memloggeruse] 
#include <chrono>
#include <iostream>

#include <be_io_filelogsheet.h>
#include <be_io_utility.h>
#include <be_system_memlog.h>

namespace BE = BiometricEvaluation;

int
main(int argc, char *argv[])
{
	auto logsheetPath = BE::IO::Utility::createTemporaryFile("memlog");
	std::shared_ptr<BE::IO::Logsheet> logsheet =
	    std::make_shared<BE::IO::FileLogsheet>("file://" + logsheetPath);
	BE::System::MemoryLogger memlog{logsheet};

	std::cout << "Starting autolog... " << std::flush;
	const auto interval = std::chrono::seconds(2);
	try {
		memlog.startAutoLogging(interval, true);
	} catch (const std::exception &e) {
		std::cerr << e.what() << '\n';
		return (EXIT_FAILURE);
	}

	//
	// Do some memory intensive work ...
	//
	std::cout << "Stopping autolog... " << std::flush;
	try {
		memlog.stopAutoLogging();
	} catch (const std::exception &e) {
		std::cerr << e.what() << '\n';
		return (EXIT_FAILURE);
	}
	std::cout << " [OKAY]\n";

	return (EXIT_SUCCESS);
}
\end{lstlisting}
